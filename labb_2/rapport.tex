\documentclass{article}
\usepackage{fixltx2e}
\usepackage[utf8]{inputenc}
\usepackage{graphicx}
\usepackage{sidecap}
\usepackage{fancyhdr}
\usepackage[margin=1.5in]{geometry}
\usepackage{amssymb,amsmath}
\usepackage[euler]{textgreek}
\usepackage[swedish]{babel}
\usepackage{mcode}


\setlength{\headheight}{21pt}

\begin{document}

% Titelsida -----------------------------
\begin{titlepage}
\begin{center}

% Upper part of the page. The '~' is needed because \\
% only works if a paragraph has started.

\textsc{\LARGE Link{\"o}pings Universitet}\\[1.5cm]


% Title
{ \huge \bfseries Modellbaserad reglering av dubbeltankar \\[0.4cm] }

% Author and supervisor
\large
\emph{Av:}\\
Hans-Filip \textsc{Elo} och Niklas \textsc{Ericson}

\vfill

% Bottom of the page
{\large \today}

\end{center}

% Slut på titelsida. ---------------------

% Innehåll ------------------------------
\newpage
\thispagestyle{empty}
\tableofcontents
\listoffigures
\listoftables

\end{titlepage}
%header ---------------------------------
\pagestyle{fancy}

\fancyhead{} % clear all header fields
\fancyhead[L]{TSRT12\slshape}
\fancyhead[R]{\today \slshape}

\fancyfoot{} % clear all footer fields
\fancyfoot[L,R]{\thepage}
\fancyfoot[L]{Modellbaserad reglering av dubbeltankar}

%slut på header ---------------------------------


\section{Inledning}
En överföringsfunktions egenskaper så som stigtid, översläng och felmarginal går att påverka genom att multiplicera funktionen med en deriverande, F\textsubscript{lead}, och en integrerande, F\textsubscript{lag}, funktion. En laboration med ett dubbeltanksystem har genomförts för att applicera teorin i verkligheten.

\section{Syfte}
Syftet med laborationen var att färdigställa en överföringsfunktion för systemet med dubbeltankar och sedan reglera systemet så det uppnår givna specifikationskrav. 

\section{Metod}

En överföringsfunktion för systemet var given enligt nedan. 

\begin{equation}
G\textsubscript{dubbel}(s) = \frac{K_{enkel1}}{sT+1} * \frac{K_{enkel2}}{sT+1} = \frac{K\textsubscript{dubbel}}{(sT+1)^2}
\end{equation}
\\
Där {\itshape T}, K\textsubscript{enkel1} och K\textsubscript{enkel2} är konstanter. K\textsubscript{dubbel} är produkten av K\textsubscript{enkel1} och K\textsubscript{enkel2}. 
\\
\\
Laborationen bestod av 2 delar. Första delen bestod av att ta fram två okända konstanter, {\itshape T} och {\itshape K}\textsubscript{dubbel}, medan den andra delen bestod i att ta fram en F\textsubscript{lead} och en F\textsubscript{lag} funktion som gav önskade egenskaper på systemet. 
\\

\subsection{Framtagning av konstanter}
Konstanterna kan bägge två lösas ut ur given överföringsfunktion för systemet, men resulterar då i en ekvation med två okända. Vi kommer alltså att behöva göra mätningar för att approximera aktuella konstanter. 

\subsubsection{Att ta fram {\itshape T}}
Konstanten {\itshape T} går att finna genom att skicka in ett steg i systemet och mäta stigtiden för utsignalen till 63\% av steget i utsignalen. Detta fås ur överföringsfunktionen för en av tankarna och laplacetransformen av ett enhetssteg (1/s) enligt nedan. 

\begin{equation}
\delta_{h1}(s) = \frac{C * K_{enkel1}}{sT+1}{(sT+1)s}
\end{equation}
\\
Där \textdelta\textsubscript{h1}(s) är skillnaden i utsignalen (vattennivå) och C är höjden på steget in. Inverstransformerar man sedan detta och sätter \textdelta\textsubscript{u}(t) = 0,63K\textsubscript{enkel}C. Löser man sedan ut {\itshape T} detta får man enligt ekvation nedan. 

\begin{equation}
T = \frac{-t}{ln{0,37}} \approx t
\end{equation}
\\
Där tiden t är tiden då utsignalen nått upp till 0,63\% av differensen mellan stationär vattennivån innan och efter stegskillnad getts i insignal. Man behöver alltså göra mätningar för att finna T. 
\\

\subsubsection{Att ta fram K\textsubscript{dubbel}}
K\textsubscript{dubbel} kan ges genom att skicka in ett steg i systemet och sedan göra mätningar vid stationärt tillstånd. Multiplicerar vi överföringsfunktionen med ett laplacetransformen för ett steg får vi. 

\begin{equation}
\delta_{h2}(s) = \frac{K_{dubbel}}{(1+s)^2}*\frac{C}{s}
\end{equation}
\\
Där \textdelta\textsubscript{h2}(s) är nivån i nedre tanken och C är spänningsdifferensen hos insignalen innan och efter steget givits. Genom att använda slutvärdessatsen kan man sedan finna slutlig nivå för systemet. 

\begin{equation}
\lim\limits_{t \to \infty}y(t) = \lim\limits_{s \to 0}sY(s)
\end{equation}
\\
Detta ger att K\textsubscript{dubbel} kan lösas ut som. 

\begin{equation}
K_{dubbel} = \frac{\delta_{h2}(s)}{C}
\end{equation}
\\

\subsection{Skapa lämpliga F\textsubscript{lead} och F\textsubscript{lag}}
För att förbättra systemets egenskaper och uppnå givna specifikationer använder vi oss av metoder som Glad och Ljung (2006) beskriver rörande kompensering med hjälp av bodediagram. Denna metod kompletteras sedan med ett egenskrivet matlabscript, vilket finns i Appendix A, för snabbare beräkningar av F\textsubscript{lead} och F\textsubscript{lag}.

\newpage

\subsection{Materiel}
Labbutrustningen som används består av en pump kopplad till ett system av två tankar. Pumpens utlopp ansluts till den övre tanken (tank 1), vars utlopp i sin tur kopplas till inloppet i den undre tanken (tank 2). Se bild nedan. 

\begin{figure}[ht!]
\centering
\includegraphics[width=70mm]{System.png}
\caption{Visualisering av labuppkopplingen}
\label{overflow}
\end{figure}

Pumpens reglersystem består av en spänningsregulator, mätsensorer för vattennivå samt programvara för PC baserad på LabView.

\subsection{Specifikationer}

\section{Resultat}
Mätningen utfördes 3 gånger med samma steg från samma arbetspunkt. Resultatet av mätningarna ges i tabell 1 nedan. Arbetspunkten var under mätningarna 10cm (1,23V). Enhetssteget ökade nivån genom att spänningsnivån höjdes till 1,28V. 

\begin{table}[ht] 
\centering 
\begin{tabular}{c c} 
Försök & T \\ [0.5ex] % inserts table heading
\hline
1 & 17 \\
2 & 18 \\
3 & 20 \\

\end{tabular} 
\caption{Tabell över mätvärden på tidskonstanten.}

\end{table}
~\\ %weird ass quickfix
Medelvärdet på T räknades ut till T=18.4 och därmed var tidskonstanten bestämd. 
Bestämningen av K\textsubscript{dubbel} gick till på liknande vis. En ekvation söktes för K\textsubscript{dubbel} och vid stationaritet kan slutvärdessatsen användas se ekvation 2. 
 
\pagebreak

Utsignalen för dubbeltankssystemet fås alltså vid stationaritet enligt ekvation 3 där u är den spänning som applicerats för att utföra steget. 

\begin{equation}
\delta_{h1}(t)=\lim\limits_{s \to 0}s*\frac{K_{dubbel}}{(1+s)^2}*\frac{u}{s}=uK_{dubbel}
\end{equation}
\\
Mätningen utfördes även här tre gånger men denna gång var steget lite olika från gång till gång. Resultatet av mätningarna ges i tabell 2 nedan. Arbetspunkten var under mätningarna 10cm (1,09V). Enhetssteget ökade nivån genom att spänningsnivån höjdes till 1,14V dvs med u=0,05V. 

\begin{table}[ht] 
\centering 
\begin{tabular}{c c} 
Försök & $\delta_{h2}$ \\ [0.5ex] % inserts table heading
\hline
1 & 1,46 \\
2 & 1,07 \\
3 & 1,36 \\

\end{tabular} 
\caption{Tabell över mätvärden på $\delta_{h2}$.}
\end{table}
~\\ %weird ass quickfix
Medelvärdet räknades ut till $\delta_{h2}=1,3$ vilket dividerat på u=0,05V ger K\textsubscript{dubbel}=25,9


\subsection{Förändring av systemets egenskaper}


\begin{figure}[ht!]
\centering
\includegraphics[width=90mm]{Test1_cut.jpg}
\caption{Stegsvaret efter kompensering med F\textsubscript{lead} och F\textsubscript{lag}.}
\label{overflow}
\end{figure}


\section{Slutsats}

\section{Källor}
Glad, Torkel och Ljung, Lennart. 2006. \textit{Reglerteknik - Grundläggande teori}. Upplaga 4:10. Lund. Studentlitteratur AB.

\newpage
\section{Appendix A}
\lstinputlisting{lab2_tankreglering.m}


\subsection{Framtagning av T och K\textsubscript{dubbel}}
I förberedelseuppgifterna till laboration visades att tidskonstanten, {\itshape T} fås av den tid det tar för stegsvaret att nå 63\% av slutvärdet. Ekvationen för ett första ordningens system ser ut som i (1) nedan. Väl i laborationen sattes en arbetspunkt på 10cm och utifrån denna applicerades en spänningsökning som motsvarar ett enhetssteg. När nivån hade nått 10,63 lästes tidskonstanten, {\itshape T} ut som den tid det tagit från att steget applicerats. 

\begin{equation}
\delta_{h1}(t)= c*K_{enkel}*(1-e^{\frac{-t}{T}})
\end{equation}
\\

\end{document}